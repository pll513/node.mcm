\documentclass[a4paper,12pt]{article}
\usepackage[framed,numbered,autolinebreaks,useliterate]{mcode}
\usepackage{CJKutf8}
\setlength{\headheight}{15pt} 
\usepackage{CJKutf8}
\usepackage{textcomp}
\usepackage{amsmath}
\usepackage{amssymb}
\usepackage{listings}
\usepackage{float}
\usepackage{xcolor}
\usepackage{color}
\usepackage{fancyhdr}
\usepackage{lastpage}
\usepackage{times}
\usepackage{mathptmx}
\usepackage{geometry}
\usepackage{booktabs}
\usepackage{graphicx}
\geometry{left=3.17cm,right=3.17cm,top=2.54cm,bottom=2.54cm}
\usepackage{indentfirst}
\setlength{\parindent}{2em}
\rhead{Page \thepage{} of \pageref{LastPage}}




\begin{document}
\begin{CJK*}{UTF8}{gbsn}

\begin{center}
\textbf{\Large{附录}}
\end{center}
\noindent\textbf{calDelayA.m}
\vspace{-15pt}
\lstset{basicstyle=\ttfamily\footnotesize,escapechar=`}
\begin{lstlisting}
function x = calDelayA(q,C,lam,qa)
mu=0.75;
L=(q/3600)/(mu-q/3600);
x = (mu-q/3600)*3600/q*(q*(1-lam)*(L/(mu*lam))-0.5*lam*C)*q/qa;

if x<0
    x=0;
end
\end{lstlisting}

\noindent\textbf{dA.m}
\vspace{-15pt}
\lstset{basicstyle=\ttfamily\footnotesize,escapechar=`}
\begin{lstlisting}
function delayA = dA(t1,t2,t3,t4)

C = t1+t2+t3+t4+4;
qa = 4694;

lam = t2/C;
q = 661;

dA1 = calDelayA(q,C,lam,qa);

lam = t1/C;
q = 1560;

dA2 = calDelayA(q,C,lam,qa);

lam = t1/C;
q = 170;

dA3 = calDelayA(q,C,lam,qa);

lam = t4/C;
q = 625;

dA4 = calDelayA(q,C,lam,qa);

lam = t3/C;
q = 802;

dA5 = calDelayA(q,C,lam,qa);

lam = t3/C;
q = 876;

dA6 = calDelayA(q,C,lam,qa);

delayA = dA1 + dA2 + dA3 + dA4 + dA5 + dA6;

\end{lstlisting}
\noindent\textbf{dB.m}
\vspace{-15pt}
\lstset{basicstyle=\ttfamily\footnotesize,escapechar=`}
\begin{lstlisting}
function delayA = dB(t1,t2,t3,t4)

C = t1+t2+t3+t4+4;
qa = 5045;

lam = t2/C;
q = 1252;

dB1 = calDelayA(q,C,lam,qa);

lam = t1/C;
q = 1458;

dB2 = calDelayA(q,C,lam,qa);

lam = t1/C;
q = 905;

dB3 = calDelayA(q,C,lam,qa);

lam = t4/C;
q = 301;

dB4 = calDelayA(q,C,lam,qa);

lam = t3/C;
q = 955;

dB5 = calDelayA(q,C,lam,qa);

lam = t3/C;
q = 174;

dB6 = calDelayA(q,C,lam,qa);

delayA = dB1 + dB2 + dB3 + dB4 + dB5 + dB6;
\end{lstlisting}

\noindent\textbf{pA.m}
\vspace{-15pt}
\lstset{basicstyle=\ttfamily\footnotesize,escapechar=`}
\begin{lstlisting}
function powerA = pA(t1,t2,t3,t4)

powerA = (661*t2 + 1560*t1 + 170*t1 + ...
625*t4 + 802*t3 + 876*t3)/3600;
\end{lstlisting}

\noindent\textbf{pB.m}
\vspace{-10pt}
\lstset{basicstyle=\ttfamily\footnotesize,escapechar=`}
\begin{lstlisting}
function powerB = pB(t1,t2,t3,t4)

powerB = (1252*t2 + 1458*t1 + 905*t1 + ...
301*t4 + 955*t3 + 174*t3)/3600;
\end{lstlisting}

\noindent\textbf{dApA.m}
\vspace{-15pt}
\lstset{basicstyle=\ttfamily\footnotesize,escapechar=`}
\begin{lstlisting}
%dApA.m
N = 1000;
c1 = 0.5;
c2 = 0.5;
Fitness1 = zeros(1,N);
Fitness2 = zeros(1,N);
pBest = zeros(4,N);
dpBest = zeros(1,N);
X = [15 + 45 * rand(1,N);15 + 45 * rand(1,N);15 + ...
45 * rand(1,N);15 + 45 * rand(1,N)];
V = [1* (rand(1,N) - 0.5);1 * (rand(1,N) - 0.5);1 ...
* (rand(1,N) - 0.5);1 * (rand(1,N) - 0.5)];
pBest1 = X;
pBest2 = X;

pA0 = pA(56,23,35,26);
dA0 = dA(56,23,35,26);

for j = 1:100
    for i = 1:N
        Fitness1(i) = dA(X(1,i),X(2,i),X(3,i),X(4,i))/dA0;
        Fitness2(i) = -pA(X(1,i),X(2,i),X(3,i),X(4,i))/pA0;
        if Fitness1(i)<dA(pBest1(1,i),pBest1(2,i),...
        pBest1(3,i),pBest1(4,i))/dA0
            pBest1(:,i) = X(:,i);
        end
        if Fitness2(i)<pA(pBest2(1,i),pBest2(2,i),...
        pBest2(3,i),pBest2(4,i))/pA0
            pBest2(:,i) = X(:,i);
        end
    end
    
    gBest1Index = find(Fitness1==min(Fitness1));
    gBest1 = X(:,gBest1Index(:,1));
    gBest2Index = find(Fitness2==min(Fitness2));
    gBest2 = X(:,gBest2Index(:,1));
    
    gBest = (gBest1+gBest2)/2;
    dgBest = sqrt((gBest1(1)-gBest2(1))^2+(gBest1(2)...
    -gBest2(2))^2+(gBest1(3)-gBest2(3))^2+(gBest1(4)-gBest2(4))^2);
    
    for i = 1:N
        dpBest(i) = sqrt((pBest1(1,i)-pBest2(1,i))^2...
        +(pBest1(2,i)-pBest2(2,i))^2+(pBest1(3,i)-...
        pBest2(3,i))^2+(pBest1(4,i)-pBest2(4,i))^2);
    end

    for i = 1:N
       if dpBest(i)<dgBest
           r = rand(1);
           if r<0.5
               pBest(:,i) = pBest1(:,i);
           else
               pBest(:,i) = pBest2(:,i);
           end
       else
           pBest(:,i) = (pBest1(:,i)+pBest2(:,i))/2;
       end
    end
    
    for i = 1:N
        r = rand(1);
        R = rand(1);
        V(:,i) = (0.5) * V(:,i) + c1 * r * ...
        (pBest(:,i)-X(:,i)) + c1 * R * (gBest-X(:,i));
        for k = 1:4
            if V(k,i)>1
                V(k,i) = 1;
            elseif V(k,i)<-1
                V(k,i) = -1;
            end    
        end
        X(:,i) = X(:,i) + V(:,i);
        for k = 1:4
            if X(k,i)>60
                X(k,i) = 60;
            elseif X(k,i)<15
                X(k,i) = 15;
            end
        end
    end
end

%test
F1 = Fitness1;
F2 = Fitness2;
plot(F1,F2,'*');
\end{lstlisting}

\noindent\textbf{dBpB.m}
\vspace{-15pt}
\lstset{basicstyle=\ttfamily\footnotesize,escapechar=`}
\begin{lstlisting}
%dBpB.m
N = 100;
c1 = 0.5;
c2 = 0.5;
Fitness1 = zeros(1,N);
Fitness2 = zeros(1,N);
pBest = zeros(4,N);
dpBest = zeros(1,N);
X = [15 + 45 * rand(1,N);15 + 45 * rand(1,N);15 + 45...
* rand(1,N);15 + 45 * rand(1,N)];
V = 10*[1* (rand(1,N) - 0.5);1 * (rand(1,N) - 0.5);1 ...
* (rand(1,N) - 0.5);1 * (rand(1,N) - 0.5)];
pBest1 = X;
pBest2 = X;

pB0 = pB(47,21,39,22);
dB0 = dB(47,21,39,22);

for j = 1:500
    for i = 1:N
        Fitness1(i) = dB(X(1,i),X(2,i),X(3,i),X(4,i))/dB0;
        Fitness2(i) = -pB(X(1,i),X(2,i),X(3,i),X(4,i))/pB0;
        if Fitness1(i)<dA(pBest1(1,i),pBest1(2,i),...
        pBest1(3,i),pBest1(4,i))/dB0
            pBest1(:,i) = X(:,i);
        end
        if Fitness2(i)<pA(pBest2(1,i),pBest2(2,i),...
        pBest2(3,i),pBest2(4,i))/pB0
            pBest2(:,i) = X(:,i);
        end
    end
    
    gBest1Index = find(Fitness1==min(Fitness1));
    gBest1 = X(:,gBest1Index(:,1));
    gBest2Index = find(Fitness2==min(Fitness2));
    gBest2 = X(:,gBest2Index(:,1));
    
    gBest = (gBest1+gBest2)/2;
    dgBest = sqrt((gBest1(1)-gBest2(1))^2+(gBest1(2)-...
    gBest2(2))^2+(gBest1(3)-gBest2(3))^2+(gBest1(4)-gBest2(4))^2);
    
    for i = 1:N
        dpBest(i) = sqrt((pBest1(1,i)-pBest2(1,i))^2...
        +(pBest1(2,i)-pBest2(2,i))^2+(pBest1(3,i)-...
        pBest2(3,i))^2+(pBest1(4,i)-pBest2(4,i))^2);
    end

    for i = 1:N
       if dpBest(i)<dgBest
           r = rand(1);
           if r<0.5
               pBest(:,i) = pBest1(:,i);
           else
               pBest(:,i) = pBest2(:,i);
           end
       else
           pBest(:,i) = (pBest1(:,i)+pBest2(:,i))/2;
       end
    end
    
    for i = 1:N
        r = rand(1);
        R = rand(1);
        V(:,i) = (0.5) * V(:,i) + c1 * r * ...
        (pBest(:,i)-X(:,i)) + c1 * R * (gBest-X(:,i));
        for k = 1:4
            if V(k,i)>1
                V(k,i) = 1;
            elseif V(k,i)<-1
                V(k,i) = -1;
            end    
        end
        X(:,i) = X(:,i) + V(:,i);
        for k = 1:4
            if X(k,i)>60
                X(k,i) = 60;
            elseif X(k,i)<15
                X(k,i) = 15;
            end
        end
    end
end

%test
F1 = Fitness1;
F2 = Fitness2;
plot(F1,F2,'*');
axis([0.8,1,-1.7,-1.2])
\end{lstlisting}

\noindent\textbf{getLostTimeFromA2B.m}
\vspace{-15pt}
\lstset{basicstyle=\ttfamily\footnotesize,escapechar=`}
\begin{lstlisting}
function lostTime = getLostTimeFromA2B(i,phi,T2,dt,t1,t2,DA)

t = i - phi + T2 * (dt/T2 - floor(dt/T2));
if t>T2
    t = t - T2;
end

if i==1||i==3
   if t>t1
       lostTime = T2 - t + DA;
   else
       lostTime = DA;
   end
elseif i==2
    if t<t1
        lostTime = t1 - t + DA;
    elseif t<t1+t2
        lostTime = DA;
    else
        lostTime = t1 + T2 - t + DA;
    end
end
\end{lstlisting} 

\noindent\textbf{findX.m}
\vspace{-15pt}
\lstset{basicstyle=\ttfamily\footnotesize,escapechar=`}
\begin{lstlisting}
clear all;clc
dApA;
X1 = X;
dBpB;
X2 = X;
clear X;
X = zeros(8,10000);
C1 = 1;
C2 = 5;

for i = 1:100
   for j = 1:100
      X(1:4,(i-1)*100+j) = X1(:,i);
      X(5:8,(i-1)*100+j) = X2(:,j);
   end
end


dab = getLostTimeFromA2B(2,8,sum(X(1:4,1))+4,60,X(5,1),X(6,1),...
calDelayA(304,sum(X(5:8,1))+4,X(5,1)/(sum(X(5:8,1))+4),520))+...
    getLostTimeFromA2B(3,8,sum(X(1:4,1))+4,60,X(5,1),X(6,1),...
    calDelayA(84,sum(X(5:8,1))+4,X(5,1)/(sum(X(5:8,1))+4),520))+...
    getLostTimeFromA2B(1,8,sum(X(1:4,1))+4,60,X(5,1),X(6,1),...
    calDelayA(132,sum(X(5:8,1))+4,X(5,1)/(sum(X(5:8,1))+4),520));
dba = getLostTimeFromA2B(2,8,sum(X(5:8,1))+4,60,X(1,1),X(2,1),...
calDelayA(1394,sum(X(1:4,1))+4,X(1,1)/(sum(X(1:4,1))+4),2495))+...
    getLostTimeFromA2B(3,8,sum(X(5:8,1))+4,60,X(1,1),X(2,1),...
    calDelayA(576,sum(X(1:4,1))+4,X(1,1)/(sum(X(1:4,1))+4),2495))...
    +getLostTimeFromA2B(1,8,sum(X(5:8,1))+4,60,X(1,1),X(2,1),...
    calDelayA(525,sum(X(1:4,1))+4,X(1,1)/(sum(X(1:4,1))+4),2495));
min = C1 * dab + C2 * dba;
X0 = X(:,1);
for i = 1:10000
    dab = getLostTimeFromA2B(2,8,sum(X(1:4,i))+4,60,X(5,i),...
    X(6,i),calDelayA(304,sum(X(5:8,i))+4,X(5,i)/(sum(X(5:8,i))+4),...
    520))+getLostTimeFromA2B(3,8,sum(X(1:4,i))+4,60,X(5,i),X(6,i),...
    calDelayA(84,sum(X(5:8,i))+4,X(5,i)/(sum(X(5:8,i))+4),520))+...
    getLostTimeFromA2B(1,8,sum(X(1:4,i))+4,60,X(5,i),X(6,i),...
    calDelayA(132,sum(X(5:8,i))+4,X(5,i)/(sum(X(5:8,1))+4),520));
    dba = getLostTimeFromA2B(2,8,sum(X(5:8,i))+4,60,X(1,i),...
    X(2,i),calDelayA(1394,sum(X(1:4,i))+4,X(1,i)/(sum(X(1:4,i))+4),...
    2495))+getLostTimeFromA2B(3,8,sum(X(5:8,i))+4,60,X(1,i),X(2,i),...
    calDelayA(576,sum(X(1:4,i))+4,X(1,i)/(sum(X(1:4,i))+4),2495))+...
    getLostTimeFromA2B(1,8,sum(X(5:8,i))+4,60,X(1,i),X(2,i),...
    calDelayA(525,sum(X(1:4,i))+4,X(1,i)/(sum(X(1:4,i))+4),2495));
    dabba = C1 * dab + C2 * dba;
    if dabba < min
        min = dabba;
        X0 = X(:,i);
    end
end

X0
\end{lstlisting} 


    
\end{CJK*}
\end{document}